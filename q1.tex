The spectrogram, depicted in figure \ref{fig:q1_spectrogram}, shows
three distinct bands of signals: 0-5kHz, 5kHz-12.5kHz and 12.5kHz-22kHz.

\begin{figure}
  \begin{center}
    \hspace*{-1in}
    \includegraphics[width=180mm]{q1_spectrogram}
    \caption{Spectrogram of the signal. \label{fig:q1_spectrogram}}
  \end{center}  
\end{figure}

Assignment instructed the use of an IIR-filter with 500Hz transition
bands and maximum of 2 dB passband ripple.  The filter specification was
drawn with {\tt speksitIIR}, obtained from the course page.  The filter
specification is depicted in figure \ref{fig:q1_filter_specification}.

XXX Chebyshev type II IIR filter.

The filtered signal's spectrogram is shown in figure
\ref{fig:q1_filtered_spectrogram}.

\begin{figure}
  \begin{center}
    \hspace*{-1in}
    \includegraphics[width=180mm]{q1_filter_specification}
    \caption{IIR filter specification for the
      passband. \label{fig:q1_filter_specification}}
  \end{center}  
\end{figure}

\begin{figure}
  \begin{center}
    \hspace*{-1in}
    \includegraphics[width=180mm]{q1_filtered_spectrogram}
    \caption{Spectrogram of the filtered signal. 
      \label{fig:q1_filtered_spectrogram}}
  \end{center}  
\end{figure}

Demodulation was done as shown in Matlab round 5.  A nice frequency to
perform the demodulation around was 11195 Hz.  The sample contains two
assignments: the song lyrics of the sample played backwards and
the calculation of ``8 * 6 * 6 * 2''.

The calculation produces the result 576.  The song lyrics are part of a
song ``Tuntematon potilas'' by Arttu Wiskari\cite{wiskari2010}.

\begin{quotation}
eikö aikani täynnä jo ois

olen jo nähnyt tämän elämän

kaiken sain ja vielä enemmän

kuule mun toive

mä haluan pois
\end{quotation}
